\documentclass[11pt, a4paper]{article}

% ── Packages ──────────────────────────────────────────────
\usepackage[utf8]{inputenc}
\usepackage[T1]{fontenc}
\usepackage{lmodern}
\usepackage[margin=1in]{geometry}
\usepackage{amsmath, amssymb}
\usepackage{graphicx}
\usepackage{xcolor}
\usepackage{listings}
\usepackage{booktabs}
\usepackage{hyperref}
\usepackage{caption}
\usepackage{float}
\usepackage{enumitem}
\usepackage{microtype}
\usepackage{fancyhdr}
\usepackage{titlesec}

% ── Colours ───────────────────────────────────────────────
\definecolor{codegreen}{rgb}{0.25, 0.50, 0.35}
\definecolor{codegray}{rgb}{0.50, 0.50, 0.50}
\definecolor{codepurple}{rgb}{0.58, 0.00, 0.82}
\definecolor{codeblue}{rgb}{0.13, 0.29, 0.53}
\definecolor{backcolour}{rgb}{0.97, 0.97, 0.97}
\definecolor{accentred}{rgb}{0.75, 0.12, 0.12}

% ── Code listing style ────────────────────────────────────
\lstdefinestyle{pythonstyle}{
    backgroundcolor=\color{backcolour},
    commentstyle=\color{codegreen}\itshape,
    keywordstyle=\color{codeblue}\bfseries,
    numberstyle=\tiny\color{codegray},
    stringstyle=\color{accentred},
    basicstyle=\ttfamily\small,
    breakatwhitespace=false,
    breaklines=true,
    captionpos=b,
    keepspaces=true,
    numbers=left,
    numbersep=8pt,
    showspaces=false,
    showstringspaces=false,
    showtabs=false,
    tabsize=4,
    frame=single,
    rulecolor=\color{codegray!40},
    framesep=4pt,
    xleftmargin=18pt,
    framexleftmargin=18pt,
}
\lstset{style=pythonstyle, language=Python}

% ── Section styling ───────────────────────────────────────
\titleformat{\section}
  {\Large\bfseries\color{codeblue}}{\thesection}{1em}{}
\titleformat{\subsection}
  {\large\bfseries\color{codeblue!80}}{\thesubsection}{1em}{}

% ── Hyperlinks ────────────────────────────────────────────
\hypersetup{
    colorlinks=true,
    linkcolor=codeblue,
    urlcolor=accentred,
    citecolor=codegreen,
}

% ── Header/footer ─────────────────────────────────────────
\pagestyle{fancy}
\fancyhf{}
\fancyhead[L]{\small\textit{When Simple Pendulums Get Complicated}}
\fancyhead[R]{\small\thepage}
\renewcommand{\headrulewidth}{0.4pt}

% ══════════════════════════════════════════════════════════
\begin{document}

% ── Title ─────────────────────────────────────────────────
\begin{center}
    {\LARGE\bfseries When Simple Pendulums Get Complicated}\\[6pt]
    {\Large\color{codegray} Why We Need Computers for Physics}\\[20pt]
    {\large Claude Opus 4.6}\\[4pt]
    {\color{codegray}\today}
\end{center}

\vspace{8pt}
\noindent\rule{\textwidth}{0.5pt}

\begin{quote}
\itshape
``The pendulum is perhaps the most studied object in all of physics.
And yet, after three centuries, it still has something to teach us
about the limits of the human mind---and the power of the machine.''
\end{quote}

\vspace{4pt}

% ══════════════════════════════════════════════════════════
\section{A mass, a string, and gravity}

Consider the humblest of physical systems: a mass $m$ suspended from
a rigid, massless rod of length $L$, free to swing in a plane under
gravity.  Let $\theta$ denote the angle the rod makes with the
downward vertical.  Applying Newton's second law along the tangential
direction gives us the \textbf{equation of motion}:

\begin{equation}
    \boxed{\;\ddot{\theta} + \frac{g}{L}\sin\theta = 0\;}
    \label{eq:full}
\end{equation}

This is a second-order, nonlinear ordinary differential equation.
It looks deceptively simple---just five symbols---but that
$\sin\theta$ hides a world of trouble.

% ══════════════════════════════════════════════════════════
\section{The textbook trick: small angles}

\subsection{The approximation}

Every introductory physics course makes the same move: assume the
angle $\theta$ is small enough that
\begin{equation}
    \sin\theta \;\approx\; \theta
    \qquad\text{(valid for } |\theta| \ll 1 \text{ rad).}
    \label{eq:smallangle}
\end{equation}
Substituting \eqref{eq:smallangle} into \eqref{eq:full} yields the
\textbf{linearised equation}:
\begin{equation}
    \ddot{\theta} + \frac{g}{L}\,\theta = 0.
    \label{eq:linear}
\end{equation}

\subsection{The exact solution}

Equation~\eqref{eq:linear} is the simple harmonic oscillator---one of
the few differential equations we can solve completely by hand.
Define the natural frequency
\begin{equation}
    \omega_0 = \sqrt{\frac{g}{L}}\,.
\end{equation}
The characteristic equation $r^2 + \omega_0^2 = 0$ has roots
$r = \pm\, i\omega_0$, so the general solution is
\begin{equation}
    \theta(t) = A\cos(\omega_0 t) + B\sin(\omega_0 t).
\end{equation}
With the natural initial conditions $\theta(0) = \theta_0$ and
$\dot{\theta}(0) = 0$ (released from rest), we obtain
\begin{equation}
    \boxed{\;\theta(t) = \theta_0 \cos\!\left(\sqrt{\frac{g}{L}}\; t\right)\;}
    \label{eq:shm_solution}
\end{equation}
and the famous period formula
\begin{equation}
    T = 2\pi\sqrt{\frac{L}{g}}\,.
    \label{eq:period}
\end{equation}

\subsection{Why this is beautiful}

Equation~\eqref{eq:shm_solution} is remarkable.  It tells us that:
\begin{itemize}[itemsep=3pt]
    \item The period $T$ depends \emph{only} on the length $L$ and gravity $g$---not
          on the mass, not on the amplitude.  This is Galileo's isochronism.
    \item The motion is a pure cosine: perfectly periodic, perfectly symmetric.
    \item We have a \emph{closed-form expression}.  Given any $t$, we can compute
          $\theta(t)$ with a pocket calculator.
\end{itemize}

This is physics at its most elegant.  One equation, one exact answer.

% ══════════════════════════════════════════════════════════
\section{When elegance breaks down}

\subsection{The honest equation}

Now suppose we push the pendulum to $\theta_0 = 90^\circ$, or $120^\circ$, or
release it from nearly inverted.  The small-angle approximation
collapses: at $\theta = \pi/2$, the error in
$\sin\theta \approx \theta$ is already $36\%$.

We are forced back to the full nonlinear equation~\eqref{eq:full}:
\[
    \ddot{\theta} + \frac{g}{L}\sin\theta = 0.
\]
Can we solve this exactly?  \emph{Sort of.}  The exact period turns
out to be
\begin{equation}
    T_{\text{exact}} = 4\sqrt{\frac{L}{g}}\; K\!\left(\sin\frac{\theta_0}{2}\right),
    \label{eq:exact_period}
\end{equation}
where $K(k)$ is the \textbf{complete elliptic integral of the first
kind}:
\begin{equation}
    K(k) = \int_0^{\pi/2} \frac{d\phi}{\sqrt{1 - k^2\sin^2\phi}}\,.
\end{equation}
This integral has no closed-form expression in terms of elementary
functions.  There is no neat formula like
Eq.~\eqref{eq:period}---just an infinite series, a table of values,
or a numerical routine.

\subsection{The deeper problem}

And it gets worse.  Even $T_{\text{exact}}$ only gives the
\emph{period}.  If we want the full trajectory $\theta(t)$ at every
instant, we need the \textbf{Jacobi elliptic functions}---objects
that most physicists have heard of but few have used by hand.

The honest truth is this: \emph{even the simplest mechanical system
resists exact, elementary solution once we remove a single
approximation.}

This is not a failure of physics.  It is a fact about nonlinear
differential equations.

% ══════════════════════════════════════════════════════════
\section{Enter the computer}

\subsection{Reformulating as a system}

What we \emph{can} do is convert Eq.~\eqref{eq:full} into a
first-order system by introducing the angular velocity
$\omega = \dot{\theta}$:
\begin{equation}
    \begin{cases}
        \;\dfrac{d\theta}{dt} = \omega \\[10pt]
        \;\dfrac{d\omega}{dt}  = -\dfrac{g}{L}\sin\theta
    \end{cases}
    \label{eq:system}
\end{equation}
This is the form that numerical integrators love.  Given an initial
state $(\theta_0, \omega_0)$, an algorithm like the classical
\textbf{Runge--Kutta method} (RK4) can march forward in time, step by
step, to any desired accuracy.

\subsection{The code}

The following Python script solves system~\eqref{eq:system} on a grid
of initial conditions and renders the \textbf{phase portrait}---a map
of all possible trajectories in the $(\theta, \omega)$ plane.

\begin{lstlisting}[caption={Phase portrait of the nonlinear pendulum.}]
import numpy as np
import matplotlib.pyplot as plt

# Parameters
g = 9.81   # gravitational acceleration (m/s^2)
L = 1.0    # pendulum length (m)

# Phase-space grid
theta = np.linspace(-2 * np.pi, 2 * np.pi, 400)
omega = np.linspace(-8, 8, 400)
THETA, OMEGA = np.meshgrid(theta, omega)

# Vector field (right-hand side of the ODE system)
dtheta_dt = OMEGA
domega_dt = -(g / L) * np.sin(THETA)
speed = np.sqrt(dtheta_dt**2 + domega_dt**2)

# Plot
fig, ax = plt.subplots(figsize=(12, 8))
strm = ax.streamplot(
    THETA, OMEGA, dtheta_dt, domega_dt,
    color=speed, cmap="coolwarm", density=2.5,
    linewidth=0.7, arrowsize=0.7,
)
fig.colorbar(strm.lines, ax=ax, label="Phase-space speed")

# Separatrix: energy E = gL (unstable equilibrium)
theta_sep = np.linspace(-2*np.pi, 2*np.pi, 1000)
omega_sep = np.sqrt(2*g/L * np.maximum(1+np.cos(theta_sep), 0))
ax.plot(theta_sep,  omega_sep, "k--", linewidth=1.8)
ax.plot(theta_sep, -omega_sep, "k--", linewidth=1.8)

ax.set_xlabel("theta (rad)")
ax.set_ylabel("omega (rad/s)")
ax.set_title("Phase Portrait: Nonlinear Pendulum")
plt.savefig("pendulum_phase_portrait.png", dpi=150)
\end{lstlisting}

% ══════════════════════════════════════════════════════════
\section{The phase portrait}

\begin{figure}[H]
    \centering
    \includegraphics[width=\textwidth]{pendulum_phase_portrait.png}
    \caption{Phase portrait of the nonlinear pendulum
             $\ddot{\theta} + (g/L)\sin\theta = 0$.
             Streamlines are coloured by speed in phase space.
             The dashed black curves are the \textbf{separatrices}.
             Green dots mark stable equilibria (pendulum hanging down);
             red dots mark unstable equilibria (pendulum inverted).}
    \label{fig:phase}
\end{figure}

\subsection{Reading the portrait}

Figure~\ref{fig:phase} encodes the complete dynamics of the pendulum
in a single image.  Three qualitatively distinct regimes are visible:

\begin{enumerate}[itemsep=4pt]
    \item \textbf{Libration} (closed orbits inside the separatrix).
          The pendulum swings back and forth without going over the top.
          Smaller loops correspond to smaller amplitudes---approaching
          simple harmonic motion near the origin.

    \item \textbf{Rotation} (wavy curves outside the separatrix).
          The pendulum has enough energy to spin continuously.
          Curves above the separatrix represent counter-clockwise
          rotation; curves below, clockwise.

    \item \textbf{The separatrix} (dashed curves).
          This is the critical boundary---the trajectory of a pendulum
          released with \emph{exactly} enough energy to reach the
          inverted position.  It would take infinite time to arrive there,
          asymptotically approaching the unstable equilibrium.
\end{enumerate}

Notice that the closed orbits are \emph{not} perfect ellipses.
In the small-angle limit, they would be---but the nonlinearity of
$\sin\theta$ distorts them, stretching the orbits as the amplitude
grows.  This distortion \emph{is} the large-angle physics that no
linearisation can capture.

% ══════════════════════════════════════════════════════════
\section{The moral}

The pendulum is a first-year physics problem, and yet its exact
behaviour lies beyond elementary mathematics.  The small-angle
solution~\eqref{eq:shm_solution} is a beautiful special case, but it
is just that: a special case.

The general solution requires either elliptic functions (an 18th-century
invention that most scientists never encounter) or numerical
computation.  The phase portrait in Figure~\ref{fig:phase} would have
taken Euler weeks of hand calculation.  A modern laptop produces it in
under a second.

This is not a story about pendulums.  It is a story about the vast
majority of differential equations in physics, engineering, and
biology.  Nonlinearity is the rule; exact solutions are the exception.
And that is why we need computers for physics.

\vfill
\begin{center}
    \small\color{codegray}
    Code and figure available at
    \url{https://github.com/VivekKarmarkar/remote-control-test}
\end{center}

\end{document}
