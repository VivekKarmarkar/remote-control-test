\documentclass[11pt, a4paper]{article}

% ── Packages ──────────────────────────────────────────────
\usepackage[utf8]{inputenc}
\usepackage[T1]{fontenc}
\usepackage{lmodern}
\usepackage[margin=1in]{geometry}
\usepackage{amsmath, amssymb}
\usepackage{graphicx}
\usepackage{xcolor}
\usepackage{listings}
\usepackage{booktabs}
\usepackage{hyperref}
\usepackage{caption}
\usepackage{float}
\usepackage{enumitem}
\usepackage{microtype}
\usepackage{fancyhdr}
\usepackage{titlesec}

% ── Colours ───────────────────────────────────────────────
\definecolor{codegreen}{rgb}{0.25, 0.50, 0.35}
\definecolor{codegray}{rgb}{0.50, 0.50, 0.50}
\definecolor{codepurple}{rgb}{0.58, 0.00, 0.82}
\definecolor{codeblue}{rgb}{0.13, 0.29, 0.53}
\definecolor{backcolour}{rgb}{0.97, 0.97, 0.97}
\definecolor{accentred}{rgb}{0.75, 0.12, 0.12}

% ── Code listing style ────────────────────────────────────
\lstdefinestyle{pythonstyle}{
    backgroundcolor=\color{backcolour},
    commentstyle=\color{codegreen}\itshape,
    keywordstyle=\color{codeblue}\bfseries,
    numberstyle=\tiny\color{codegray},
    stringstyle=\color{accentred},
    basicstyle=\ttfamily\small,
    breakatwhitespace=false,
    breaklines=true,
    captionpos=b,
    keepspaces=true,
    numbers=left,
    numbersep=8pt,
    showspaces=false,
    showstringspaces=false,
    showtabs=false,
    tabsize=4,
    frame=single,
    rulecolor=\color{codegray!40},
    framesep=4pt,
    xleftmargin=18pt,
    framexleftmargin=18pt,
}
\lstset{style=pythonstyle, language=Python}

% ── Section styling ───────────────────────────────────────
\titleformat{\section}
  {\Large\bfseries\color{codeblue}}{\thesection}{1em}{}
\titleformat{\subsection}
  {\large\bfseries\color{codeblue!80}}{\thesubsection}{1em}{}

% ── Hyperlinks ────────────────────────────────────────────
\hypersetup{
    colorlinks=true,
    linkcolor=codeblue,
    urlcolor=accentred,
    citecolor=codegreen,
}

% ── Header/footer ─────────────────────────────────────────
\pagestyle{fancy}
\fancyhf{}
\fancyhead[L]{\small\textit{When Simple Pendulums Get Complicated}}
\fancyhead[R]{\small\thepage}
\renewcommand{\headrulewidth}{0.4pt}

% ══════════════════════════════════════════════════════════
\begin{document}

% ── Title ─────────────────────────────────────────────────
\begin{center}
    {\LARGE\bfseries When Simple Pendulums Get Complicated}\\[6pt]
    {\Large\color{codegray} Why We Need Computers for Physics}\\[20pt]
    {\large Claude Opus 4.6}\\[4pt]
    {\color{codegray}\today}
\end{center}

\vspace{8pt}
\noindent\rule{\textwidth}{0.5pt}

% ── Abstract ──────────────────────────────────────────────
\begin{abstract}
\noindent
The simple pendulum is among the most studied systems in classical
mechanics, yet its exact large-angle dynamics lie beyond elementary
mathematics.  We derive the equation of motion from first principles,
solve it analytically under the small-angle approximation, and then
show how that solution breaks down at large amplitudes.  Using energy
conservation, we derive the exact period as an elliptic integral and
the separatrix that divides oscillation from rotation.  A fourth-order
Runge--Kutta integrator is implemented in Python to produce
time-series comparisons, period-versus-amplitude curves, and a
complete phase portrait.  The pendulum serves as a case study for a
universal theme: nonlinearity is the rule in physics, and computation
is the tool that lets us confront it.
\end{abstract}

\vspace{6pt}

\begin{quote}
\itshape
``The pendulum is perhaps the most studied object in all of physics.
And yet, after three centuries, it still has something to teach us
about the limits of the human mind---and the power of the machine.''
\end{quote}

\vspace{4pt}

% ══════════════════════════════════════════════════════════
\section{A mass, a string, and gravity}
\label{sec:setup}

Consider the humblest of physical systems: a mass $m$ suspended from
a rigid, massless rod of length $L$, free to swing in a plane under
gravity $g$.  Let $\theta$ denote the angle the rod makes with the
downward vertical.

The tangential component of the gravitational force on the mass is
$F_\tau = -mg\sin\theta$ (the minus sign because it acts to restore
the pendulum toward $\theta = 0$).  The tangential acceleration is
$a_\tau = L\ddot{\theta}$.  Newton's second law, $F_\tau = m\,a_\tau$,
gives~\cite{taylor}
\[
    mL\ddot{\theta} = -mg\sin\theta,
\]
which simplifies to the \textbf{equation of motion}:
\begin{equation}
    \boxed{\;\ddot{\theta} + \frac{g}{L}\sin\theta = 0\;}
    \label{eq:full}
\end{equation}
This is a second-order, nonlinear ordinary differential equation.
It looks deceptively simple---just five symbols---but that
$\sin\theta$ hides a world of trouble.

% ══════════════════════════════════════════════════════════
\section{The textbook trick: small angles}
\label{sec:smallangle}

\subsection{The approximation}

Every introductory physics course makes the same move: assume the
angle $\theta$ is small enough that
\begin{equation}
    \sin\theta \;\approx\; \theta
    \qquad\text{(valid for } |\theta| \ll 1 \text{ rad).}
    \label{eq:smallangle}
\end{equation}
But how small is ``small enough''?  Figure~\ref{fig:sinerror} shows
the relative error of this approximation.  At $14^\circ$ the error
reaches $1\%$; at $31^\circ$ it hits $5\%$; and by $43^\circ$ the
error exceeds $10\%$.  The approximation is far more fragile than
most textbooks admit.

\begin{figure}[H]
    \centering
    \includegraphics[width=\textwidth]{fig_sin_error.png}
    \caption{Relative error of the small-angle approximation
             $\sin\theta \approx \theta$.  The inset shows
             $\sin\theta$ (blue) diverging from $\theta$ (red dashed)
             as the angle grows.  Vertical lines mark the $1\%$,
             $5\%$, and $10\%$ error thresholds.}
    \label{fig:sinerror}
\end{figure}

Substituting~\eqref{eq:smallangle} into~\eqref{eq:full} yields the
\textbf{linearised equation}:
\begin{equation}
    \ddot{\theta} + \frac{g}{L}\,\theta = 0.
    \label{eq:linear}
\end{equation}

\subsection{The exact solution}

Equation~\eqref{eq:linear} is the simple harmonic oscillator---one of
the few differential equations we can solve completely by hand.
Define the natural frequency
$\omega_0 = \sqrt{g/L}$.
The characteristic equation $r^2 + \omega_0^2 = 0$ has purely
imaginary roots $r = \pm\, i\omega_0$, which yield the general
solution
\begin{equation}
    \theta(t) = A\cos(\omega_0 t) + B\sin(\omega_0 t).
\end{equation}
With the initial conditions $\theta(0) = \theta_0$ and
$\dot{\theta}(0) = 0$ (released from rest), we find $A = \theta_0$,
$B = 0$, giving
\begin{equation}
    \boxed{\;\theta(t) = \theta_0 \cos\!\left(\sqrt{\frac{g}{L}}\; t\right)\;}
    \label{eq:shm_solution}
\end{equation}
and the famous period formula
\begin{equation}
    T_0 = \frac{2\pi}{\omega_0} = 2\pi\sqrt{\frac{L}{g}}\,.
    \label{eq:period}
\end{equation}

\subsection{Why this is beautiful}

Equation~\eqref{eq:shm_solution} tells us that:
\begin{itemize}[itemsep=3pt]
    \item The period $T_0$ depends \emph{only} on $L$ and $g$---not
          on the mass, not on the amplitude.  This is
          \emph{Galileo's isochronism}.
    \item The motion is a pure cosine: perfectly periodic, perfectly
          symmetric.
    \item We have a \emph{closed-form expression}.  Given any $t$, we
          can compute $\theta(t)$ with a pocket calculator.
\end{itemize}
This is physics at its most elegant.  One equation, one exact
answer~\cite{taylor}.

% ══════════════════════════════════════════════════════════
\section{When elegance breaks down}
\label{sec:breakdown}

\subsection{The honest equation}

Now suppose we push the pendulum to $\theta_0 = 90^\circ$, or
$120^\circ$, or release it from nearly inverted.  The small-angle
approximation collapses.

We are forced back to the full nonlinear
equation~\eqref{eq:full}.  Can we make progress?  Yes---but only by
bringing in a conserved quantity that the linearised analysis never
needed.

\subsection{Energy: the hidden conserved quantity}
\label{sec:energy}

Multiply the equation of motion~\eqref{eq:full} by
$\dot{\theta}$:
\[
    \dot{\theta}\,\ddot{\theta}
    + \frac{g}{L}\,\dot{\theta}\sin\theta = 0.
\]
The left side is an exact time derivative~\cite{goldstein}:
\[
    \frac{d}{dt}\!\left[
        \tfrac{1}{2}\dot{\theta}^{\,2}
        - \frac{g}{L}\cos\theta
    \right] = 0.
\]
Therefore the quantity in brackets is constant.  Defining the
\textbf{total energy per unit} $mL^2$ as
\begin{equation}
    \boxed{\;
    \mathcal{E}
    = \tfrac{1}{2}\,\dot{\theta}^{\,2}
      + \frac{g}{L}\bigl(1 - \cos\theta\bigr)
    = \text{const},\;}
    \label{eq:energy}
\end{equation}
where the constant $(g/L)(1-\cos\theta)$ is chosen so that
$\mathcal{E} = 0$ at the stable equilibrium
$(\theta, \dot\theta) = (0, 0)$.

\medskip
\noindent
\textbf{The separatrix.}\;
At the unstable equilibrium $\theta = \pi$ (pendulum balanced
inverted), the potential energy is
$\mathcal{E}_{\text{sep}} = 2g/L$.  The \textbf{separatrix} is the
set of all states with exactly this critical energy:
\begin{equation}
    \tfrac{1}{2}\dot{\theta}^{\,2}
    + \frac{g}{L}(1 - \cos\theta) = \frac{2g}{L},
\end{equation}
which gives
\begin{equation}
    \boxed{\;
    \dot{\theta} = \pm\sqrt{\frac{2g}{L}(1 + \cos\theta)}
    = \pm\, 2\sqrt{\frac{g}{L}}\;\cos\frac{\theta}{2}\;.
    \;}
    \label{eq:separatrix}
\end{equation}
States below this energy oscillate (libration); states above it
spin continuously (rotation).

\subsection{The exact period: from energy to elliptic integrals}
\label{sec:elliptic}

Energy conservation lets us derive the exact period.  For a pendulum
released from rest at angle $\theta_0$, we have
$\mathcal{E} = (g/L)(1 - \cos\theta_0)$.  Solving
Eq.~\eqref{eq:energy} for $\dot{\theta}$:
\begin{equation}
    \dot{\theta}
    = \sqrt{\frac{2g}{L}\bigl(\cos\theta - \cos\theta_0\bigr)}\,.
    \label{eq:thetadot_energy}
\end{equation}
Separating variables, the time for a quarter-period (from
$\theta = 0$ to $\theta = \theta_0$) is
\begin{equation}
    \frac{T}{4}
    = \sqrt{\frac{L}{2g}}
      \int_0^{\theta_0}
      \frac{d\theta}
           {\sqrt{\cos\theta - \cos\theta_0}}\,.
    \label{eq:quarter_period}
\end{equation}
Using the substitution $\sin(\theta/2) = k\sin\phi$ with
$k = \sin(\theta_0/2)$, the integral transforms into~\cite{goldstein}
\begin{equation}
    \boxed{\;
    T(\theta_0)
    = \frac{4}{\omega_0}\,K(k),
    \qquad
    k = \sin\frac{\theta_0}{2},
    \;}
    \label{eq:exact_period}
\end{equation}
where $K(k)$ is the \textbf{complete elliptic integral of the first
kind}:
\begin{equation}
    K(k) = \int_0^{\pi/2}
           \frac{d\phi}{\sqrt{1 - k^2\sin^2\phi}}\,.
    \label{eq:elliptic_K}
\end{equation}
In the small-angle limit ($k \to 0$), $K(0) = \pi/2$, and
Eq.~\eqref{eq:exact_period} recovers the SHM period
$T_0 = 2\pi/\omega_0$.  As $\theta_0 \to \pi$, $k \to 1$ and
$K(k) \to \infty$: the period diverges.  The integrand in
Eq.~\eqref{eq:elliptic_K} has no antiderivative in elementary
functions---this is where pencil-and-paper physics reaches its limit.

\subsection{The deeper problem}

Even the exact period formula only gives one number.  If we want the
full trajectory $\theta(t)$ at every instant, we need the
\textbf{Jacobi elliptic functions}~\cite{goldstein}---objects that
most physicists have heard of but few have used by hand.

The honest truth is this: \emph{even the simplest mechanical system
resists exact, elementary solution once we remove a single
approximation.}

% ══════════════════════════════════════════════════════════
\section{The geometry of dynamics}
\label{sec:phase}

\subsection{What is phase space?}

Define the angular velocity $\omega = \dot{\theta}$.  The state of
the pendulum at any instant is fully described by the pair
$(\theta, \omega)$.  The set of all such pairs forms the
\textbf{phase plane}.

As time evolves, the state traces a curve---called a
\textbf{trajectory} or \textbf{orbit}---through the phase plane.
A \textbf{phase portrait} is the complete map of all possible
trajectories, one for every initial condition.  It is a snapshot of
the system's entire dynamical repertoire.

\subsection{Equilibria and their stability}
\label{sec:stability}

An equilibrium is a state where $\dot{\theta} = \ddot{\theta} = 0$,
i.e., $\sin\theta = 0$.  This gives
$\theta^* = 0$ (hanging down) and $\theta^* = \pi$ (inverted).

To classify their stability~\cite{strogatz}, we linearise the system
$(\dot{\theta}, \dot{\omega}) = (\omega,\, -(g/L)\sin\theta)$
around each fixed point.  The Jacobian is
\[
    J = \begin{pmatrix}
        0 & 1 \\[4pt]
        -\dfrac{g}{L}\cos\theta^* & 0
    \end{pmatrix}.
\]

\medskip
\noindent
\textbf{At $\theta^* = 0$ (hanging down):}\;
$J$ has eigenvalues $\lambda = \pm\, i\omega_0$.  Pure imaginary
eigenvalues indicate a \textbf{centre}: nearby orbits form closed
loops.  The pendulum oscillates.

\medskip
\noindent
\textbf{At $\theta^* = \pi$ (inverted):}\;
$\cos\pi = -1$, so $J$ has eigenvalues
$\lambda = \pm\sqrt{g/L}$---one positive, one negative.  This is a
\textbf{saddle point}: trajectories approach along one direction and
flee along the other.  The separatrix~\eqref{eq:separatrix} is the
stable manifold of this saddle.

\subsection{The three regimes}

Energy conservation and stability analysis together explain the
structure of the phase portrait.  Table~\ref{tab:regimes} summarises
the three qualitatively distinct regimes.

\begin{table}[H]
\centering
\caption{Dynamical regimes of the nonlinear pendulum.}
\label{tab:regimes}
\begin{tabular}{@{}llll@{}}
\toprule
\textbf{Regime} & \textbf{Energy} & \textbf{Phase portrait}
    & \textbf{Physical motion} \\
\midrule
Libration
    & $\mathcal{E} < 2g/L$
    & Closed orbits around $\theta = 0$
    & Swings back and forth \\[4pt]
Separatrix
    & $\mathcal{E} = 2g/L$
    & Homoclinic orbit through $\theta = \pm\pi$
    & Asymptotically reaches inverted \\[4pt]
Rotation
    & $\mathcal{E} > 2g/L$
    & Open, wavy curves
    & Continuous spinning \\
\bottomrule
\end{tabular}
\end{table}

% ══════════════════════════════════════════════════════════
\section{Enter the computer}
\label{sec:computation}

\subsection{Reformulating as a system}

We convert Eq.~\eqref{eq:full} into a first-order system with
$\omega = \dot{\theta}$:
\begin{equation}
    \begin{cases}
        \;\dfrac{d\theta}{dt} = \omega \\[10pt]
        \;\dfrac{d\omega}{dt}  = -\dfrac{g}{L}\sin\theta
    \end{cases}
    \label{eq:system}
\end{equation}
Given an initial state $(\theta_0, \omega_0)$, we march forward in
time using the classical \textbf{Runge--Kutta method}
(RK4)~\cite{press}.  The
core step advances the state by one time increment $h$:

\begin{lstlisting}[caption={The RK4 integration step.}]
def rk4_step(f, state, t, h):
    """Single 4th-order Runge-Kutta step."""
    k1 = h * f(state, t)
    k2 = h * f(state + k1/2, t + h/2)
    k3 = h * f(state + k2/2, t + h/2)
    k4 = h * f(state + k3, t + h)
    return state + (k1 + 2*k2 + 2*k3 + k4) / 6
\end{lstlisting}

Each evaluation of $f$ computes $(\omega,\, -(g/L)\sin\theta)$ from
system~\eqref{eq:system}.  The four slopes $k_1, \ldots, k_4$ are
combined to cancel lower-order error terms, giving $O(h^5)$ local
truncation error per step.

\subsection{When does the approximation fail?}

We integrate system~\eqref{eq:system} from rest at three initial
angles---$10^\circ$, $90^\circ$, and $170^\circ$---and overlay the
numerical trajectory with the SHM prediction
$\theta(t) = \theta_0\cos(\omega_0 t)$.

\begin{figure}[H]
    \centering
    \includegraphics[width=\textwidth]{fig_trajectory_comparison.png}
    \caption{Small-angle analytical solution (red dashed) versus
             nonlinear RK4 solution (blue solid) for three initial
             amplitudes.  At $10^\circ$ the curves are nearly
             identical.  At $90^\circ$ the nonlinear period is visibly
             longer.  At $170^\circ$ the solutions diverge
             dramatically---the real pendulum oscillates far more
             slowly than the harmonic prediction.  Shaded purple
             regions highlight the growing discrepancy.}
    \label{fig:trajectories}
\end{figure}

The message is clear: the small-angle solution is excellent for
gentle swings, but it becomes qualitatively wrong as amplitude
increases.

\subsection{Period versus amplitude}

The exact period~\eqref{eq:exact_period} depends on the initial
amplitude $\theta_0$ through the elliptic integral.
Figure~\ref{fig:period} shows the ratio
$T(\theta_0)/T_0$ as a function of amplitude, with the exact
elliptic-integral curve, a quadratic approximation
$1 + \theta_0^2/16$, and numerical measurements from our RK4
integrator.

\begin{figure}[H]
    \centering
    \includegraphics[width=\textwidth]{fig_period_ratio.png}
    \caption{Period ratio $T(\theta_0)/T_0$ as a function of initial
             amplitude.  The blue curve is the exact result from
             Eq.~\eqref{eq:exact_period}; red dots are RK4
             measurements; the green dashed line is the quadratic
             approximation.  At $30^\circ$ the period exceeds the SHM
             prediction by $1.7\%$; at $90^\circ$ by $18\%$; near
             $180^\circ$ it diverges to infinity.  Galileo's
             isochronism is a small-angle illusion.}
    \label{fig:period}
\end{figure}

\subsection{The phase portrait}

Finally, we compute the vector field
$(d\theta/dt,\, d\omega/dt)$ on a grid covering
$\theta \in [-2\pi, 2\pi]$, $\omega \in [-8, 8]$ and render the
streamlines.  The separatrix~\eqref{eq:separatrix} is overlaid
as a dashed curve; equilibria are marked with coloured dots.

\begin{figure}[H]
    \centering
    \includegraphics[width=\textwidth]{pendulum_phase_portrait.png}
    \caption{Phase portrait of the nonlinear pendulum.
             Streamlines are coloured by speed in phase space.
             The dashed black curves are the
             \textbf{separatrices}~\eqref{eq:separatrix}.
             Green dots mark stable equilibria (centres at
             $\theta = 0, \pm 2\pi$); red dots mark unstable
             equilibria (saddle points at $\theta = \pm\pi$).
             Closed orbits inside the separatrix correspond to
             libration; open curves outside correspond to rotation.
             See Table~\ref{tab:regimes}.}
    \label{fig:phase}
\end{figure}

Notice that the closed orbits near the origin are nearly circular
(approaching the SHM ellipses), while those near the separatrix are
visibly flattened---a direct signature of the nonlinearity.

% ══════════════════════════════════════════════════════════
\section{The moral}
\label{sec:moral}

The pendulum is a first-year physics problem, and yet its exact
behaviour lies beyond elementary mathematics.  The small-angle
solution~\eqref{eq:shm_solution} is a beautiful special case, but it
is just that: a special case.

The general solution requires either elliptic functions (an
18th-century invention that most scientists never encounter) or
numerical computation.  The phase portrait in
Figure~\ref{fig:phase} would have taken Euler weeks of hand
calculation.  A modern laptop produces it in under a second.

\medskip
\noindent
\textbf{Beyond the ideal.}\;
We have studied the conservative pendulum: no friction, no driving
force.  In reality, damping causes every trajectory to spiral inward
toward the hanging equilibrium.  More dramatically, adding a periodic
driving force $A\cos(\Omega t)$ to Eq.~\eqref{eq:full} produces
\emph{chaos}---sensitive dependence on initial conditions, fractal
basins of attraction, and period-doubling cascades~\cite{strogatz}.
The simple pendulum, pushed a little further, becomes one of the
canonical models of nonlinear dynamics.

\medskip
This is not a story about pendulums.  It is a story about the vast
majority of differential equations in physics, engineering, and
biology.  Nonlinearity is the rule; exact solutions are the exception.
And that is why we need computers for physics.

\vfill

% ══════════════════════════════════════════════════════════
\begin{thebibliography}{9}

\bibitem{strogatz}
S.~H.~Strogatz,
\textit{Nonlinear Dynamics and Chaos},
2nd ed.\ (Westview Press, 2015).

\bibitem{taylor}
J.~R.~Taylor,
\textit{Classical Mechanics},
(University Science Books, 2005).

\bibitem{goldstein}
H.~Goldstein, C.~Poole, and J.~Safko,
\textit{Classical Mechanics},
3rd ed.\ (Addison-Wesley, 2002).

\bibitem{press}
W.~H.~Press, S.~A.~Teukolsky, W.~T.~Vetterling, and B.~P.~Flannery,
\textit{Numerical Recipes},
3rd ed.\ (Cambridge University Press, 2007).

\bibitem{scipy}
P.~Virtanen \textit{et al.},
``SciPy 1.0: fundamental algorithms for scientific computing in
Python,''
\textit{Nature Methods} \textbf{17}, 261 (2020).

\end{thebibliography}

\vspace{12pt}
\begin{center}
    \small\color{codegray}
    All code and figures available at
    \url{https://github.com/VivekKarmarkar/remote-control-test}
\end{center}

% ══════════════════════════════════════════════════════════
\clearpage
\appendix
\section{Complete Python code}
\label{app:code}

The following scripts reproduce all figures in this report.  They
require \texttt{numpy}, \texttt{matplotlib}, and
\texttt{scipy}~\cite{scipy}.

\subsection{Core integrator (\texttt{rk4\_pendulum.py})}

\begin{lstlisting}[caption={RK4 integrator and utility functions.}]
import numpy as np

g, L = 9.81, 1.0   # gravitational acceleration, pendulum length

def pendulum_ode(state, _t, g=g, L=L):
    theta, omega = state
    return np.array([omega, -(g / L) * np.sin(theta)])

def rk4_step(f, state, t, h, *args, **kwargs):
    k1 = h * f(state, t, *args, **kwargs)
    k2 = h * f(state + k1/2, t + h/2, *args, **kwargs)
    k3 = h * f(state + k2/2, t + h/2, *args, **kwargs)
    k4 = h * f(state + k3, t + h, *args, **kwargs)
    return state + (k1 + 2*k2 + 2*k3 + k4) / 6

def integrate(f, state0, t_span, h, *args, **kwargs):
    t_start, t_end = t_span
    t_vals = np.arange(t_start, t_end + h/2, h)
    states = np.zeros((len(t_vals), len(state0)))
    states[0] = state0
    for i in range(1, len(t_vals)):
        states[i] = rk4_step(f, states[i-1], t_vals[i-1],
                              h, *args, **kwargs)
    return t_vals, states

def compute_energy(theta, omega, g=g, L=L):
    return 0.5 * omega**2 + (g/L) * (1 - np.cos(theta))

def measure_period(t, theta):
    crossings = []
    for i in range(1, len(theta)):
        if theta[i-1] > 0 and theta[i] <= 0:
            t_cross = t[i-1] - theta[i-1] * (t[i] - t[i-1]) \
                      / (theta[i] - theta[i-1])
            crossings.append(t_cross)
    if len(crossings) < 2:
        return np.inf
    return np.mean(np.diff(crossings))
\end{lstlisting}

\subsection{Phase portrait (\texttt{pendulum\_phase\_portrait.py})}

\begin{lstlisting}[caption={Phase portrait with separatrix and equilibria.}]
import numpy as np
import matplotlib.pyplot as plt

g, L = 9.81, 1.0
theta = np.linspace(-2*np.pi, 2*np.pi, 400)
omega = np.linspace(-8, 8, 400)
THETA, OMEGA = np.meshgrid(theta, omega)

dtheta_dt = OMEGA
domega_dt = -(g / L) * np.sin(THETA)
speed = np.sqrt(dtheta_dt**2 + domega_dt**2)

fig, ax = plt.subplots(figsize=(12, 8))
strm = ax.streamplot(THETA, OMEGA, dtheta_dt, domega_dt,
    color=speed, cmap="coolwarm", density=2.5,
    linewidth=0.7, arrowsize=0.7)
fig.colorbar(strm.lines, ax=ax, label="Phase-space speed")

# Separatrix: E = 2g/L (energy at unstable equilibrium)
theta_sep = np.linspace(-2*np.pi, 2*np.pi, 1000)
omega_sep = np.sqrt(2*g/L * np.maximum(1+np.cos(theta_sep), 0))
ax.plot(theta_sep,  omega_sep, "k--", linewidth=1.8,
        label="Separatrix")
ax.plot(theta_sep, -omega_sep, "k--", linewidth=1.8)

# Equilibrium points
for n in range(-2, 3):
    if n % 2 == 0:
        ax.plot(n*np.pi, 0, "o", color="#2ecc71", markersize=9,
                zorder=5, label="Stable eq." if n == 0 else "")
    else:
        ax.plot(n*np.pi, 0, "o", color="#e74c3c", markersize=9,
                zorder=5, label="Unstable eq." if n == 1 else "")

ax.set_xlabel(r"$\theta$ (rad)", fontsize=14)
ax.set_ylabel(r"$\omega$ (rad/s)", fontsize=14)
ax.set_xticks([-2*np.pi, -np.pi, 0, np.pi, 2*np.pi])
ax.set_xticklabels([r"$-2\pi$", r"$-\pi$", "0",
                     r"$\pi$", r"$2\pi$"], fontsize=12)
ax.legend(fontsize=11, loc="upper right")
plt.tight_layout()
plt.savefig("pendulum_phase_portrait.png", dpi=150)
\end{lstlisting}

\end{document}
